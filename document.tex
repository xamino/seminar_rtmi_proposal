\documentclass{acm_proc_article-sp}

% UTF8 encoding and scalable fonts
\usepackage[utf8]{inputenc}
\usepackage[T1]{fontenc}

\usepackage{microtype}
\usepackage{graphicx}
\usepackage{subfigure}
\usepackage{booktabs}
\usepackage{listings}
\usepackage{url}
\usepackage[hidelinks]{hyperref}

% \usepackage{doi}
% \setlength{\paperheight}{11.69in}

\begin{document}

% Leave as is
\conferenceinfo{5th Seminar on Research Trends in Media Informatics (RTMI '13).}{\\February 2013, Ulm University, Ulm, Germany.}
\CopyrightYear{2013}

\title{Interaction Issues during Autonomous Driving}

\numberofauthors{1}
\author{
\alignauthor
Tamino P.S.M. Hartmann\\
       \affaddr{Institute of Media Informatics}\\
       \affaddr{Ulm University}\\
       \affaddr{Ulm, Germany}\\
       \email{tamino.hartmann@uni-ulm.de}
}

\maketitle
\begin{abstract}

The following paper takes a look at interaction issues that can and have arisen for autonomous vehicles.
After a look at the existing state of the industry, we will compare technologies already in use and how they handle possible issues.
We will then try to extrapolate the current state of interaction for autonomous cars into the future.

This proposal will take a more in-close look at what is to be considered when facing interaction issues between the system and its driver.
To put that into perspective, we will begin by taking a look at existing autonomous vehicles and what can already be deduced from them.
This could include driverless trains and autopilots of airplanes.

\end{abstract}

\keywords{Autonomous vehicles, self-driving cars, interaction}

\section{Introduction}

Autonomous vehicles have a surprisingly long history, although they have only recently become feasible.

This section will present a general history of all autonomous vehicles that have been envision, built, and used.
Thus we will offer the current state of the technology as we understand, laying the basic groundwork necessary for our work.
It will also serve to define the scope of our work – namely that we will primarily target our work at self-driving cars such as the current Google car \cite{www:google_car}.

\section{Current State}

\subsection{Current Issues}

driver distraction accounted for more than 80 percent of “safety-critical events.

\subsection{Current Technology}

Although most companies that have begun developing autonomous car technology remain relatively secretive, some information is publicly available.
Therefore we will take a look at systems that are known to exist at this time, including the few systems already commercially available.
We will briefly highlight their technological standing and what can be deduced for interaction issues from how they have implemented the technology.

\subsubsection{Autonomous Trains and Airplanes}

\subsubsection{Daimler / Mercedes-Benz}

\subsubsection{BMW}

\subsubsection{Google}

\subsubsection{Toyota}

\subsubsection{Audi}

\section{Driver Alertness}

The most basic aspect that has to be considered for autonomous vehicle interaction is that its passengers will require interaction.
That will range from simply telling the car where to go to passing control to a human driver for handling of situations that the computer can not safely understand.

However one cannot simply have control constantly switching between the machine and the driver – the driver might not be alert, might not be allowed to take control, or a multitude of other reasons.
For this reason technology such as the Attention Assistance function of a S-class Mercedes can be significant for interaction issues.
Instead of only having the vehicle read and process its surroundings, it can and probably should also monitor the driver and or its passengers.

\section{Cultural and Social Issues}

take a look at laws, practices, etc that need to be considered.

\subsection{Driver Acceptance}

Should we keep the driver in the loop?
If yes, how?

\section{Primary Issue: the Hand-off}

The most dangerous moment in a self-driving car involves no immediate or obvious peril.
It is not when, say, the computer must avoid a vehicle swerving into its lane or navigate some other recognizable hazard of the road -- a patch of ice, or a clueless pedestrian stepping into traffic.
It is when something much more routine takes place: The computer hands over control of the vehicle to a human being.
In that instant, the human must quickly rouse herself from whatever else she might have been doing while the computer handled the car and focus her attention on the road.
As scientists now studying this moment have come to realize, the hand-off is laden with risks.
From \cite{www:huffington_post}.

This brings up the most challenging obstacle on our road to the autonomous- driving future: managing the handoff.
For as long as anyone, even Google, is willing to predict, cars will by necessity be semiautonomous; human drivers will still have to play some role.
But figuring out what that role will be is complicated.
Are we pilots or copilots? How far out of the loop can we be taken?
From \cite{www:wired}.

\section{Secondary Issues}

Distractions because of free time, low alertness, etc.

\section{Conclusion}

TODO

% do not change the bibliography style
\bibliographystyle{abbrv}
\bibliography{sources}  

\balancecolumns

\end{document}
