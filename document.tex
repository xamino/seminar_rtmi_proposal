\documentclass{acm_proc_article-sp}

% UTF8 encoding and scalable fonts
\usepackage[utf8]{inputenc}
\usepackage[T1]{fontenc}

\usepackage{microtype}
\usepackage{graphicx}
\usepackage{subfigure}
\usepackage{booktabs}
\usepackage{listings}
\usepackage{url}

% \usepackage{doi}
% \setlength{\paperheight}{11.69in}

% Comment in when writing in German
%\usepackage{ngerman}

\begin{document}

% Leave as is
\conferenceinfo{5th Seminar on Research Trends in Media Informatics (RTMI '13).}{\\February 2013, Ulm University, Ulm, Germany.}
\CopyrightYear{2013} 

\title{Interaction Issues during Autonomous Driving}

\numberofauthors{1} 
\author{
\alignauthor
Tamino P.S.M. Hartmann\\
       \affaddr{Institute of Media Informatics}\\
       \affaddr{Ulm University}\\
       \affaddr{Ulm, Germany}\\
       \email{tamino.hartmann@uni-ulm.de}
}

\maketitle
\begin{abstract}
The abstract should summarize the complete paper and give a specific overview of what is covered in the paper. Do not confuse the abstract with an introduction.
\end{abstract}

% See http://www.acm.org/about/class/ on how to select appropriate categories, terms, and keywords
% Use the 1998 classification for now.
% A category with the (minimum) three required fields
\category{H.4}{Information Systems Applications}{Miscellaneous}
%A category including the fourth, optional field follows...
\category{D.2.8}{Software Engineering}{Metrics}[complexity measures, performance measures]
% Select applicable general terms
\terms{Theory} 

% Specify relevant keywords (not more than 5)
\keywords{ACM proceedings, \LaTeX, text tagging} 

\section{Introduction}
Please have a look at the ACM sample file and FAQs at \url{acm.org/sigs/publications/proceedings-templates} for formatting help, concerning figures, tables, and math. Examples of Table and Figure environments are included as comments below.

\subsection{A Subsection}

\subsubsection{A subsubsection}
Usually three hierarchy levels are sufficient for papers. You should rethink your structure, if you need more levels.

\begin{table}
	\centering
	\caption{Frequency of Special Characters}
	\label{tab:table1}
	\begin{tabular}{|c|c|l|} \hline
	Non-English or Math&Frequency&Comments\\ \hline
	\O & 1 in 1,000& For Swedish names\\ \hline
	$\pi$ & 1 in 5& Common in math\\ \hline
	\$ & 4 in 5 & Used in business\\ \hline
	$\Psi^2_1$ & 1 in 40,000& Unexplained usage\\
	\hline\end{tabular}
\end{table}

Let us put some more text here to test the cite stuff \cite{Lamport:LaTeX}.

\begin{table*} % use the table* version to create tables spanning both columns
	\centering
	\caption{Some Typical Commands}
	\label{tab:table2}
	\begin{tabular}{|c|c|l|} \hline
	Command&A Number&Comments\\ \hline
	\texttt{{\char'134}alignauthor} & 100& Author alignment\\ \hline
	\texttt{{\char'134}numberofauthors}& 200& Author enumeration\\ \hline
	\texttt{{\char'134}table}& 300 & For tables\\ \hline
	\texttt{{\char'134}table*}& 400& For wider tables\\ \hline\end{tabular}
\end{table*}

%\begin{figure}
%	\centering
%	\includegraphics[width=\columnwidth]{fly.pdf}
%	\caption{A fly}
%	\label{fig:fly}
%\end{figure}

\begin{displaymath}
\lim_{x\rightarrow\infty} \frac{f(x)}{g(x)} = L.
\end{displaymath}


% do not change the bibliography style
\bibliographystyle{abbrv}
% sigproc.bib is the name of the Bibliography in this case, change to your file
\bibliography{sources}  
% You must have a proper ".bib" file
%  and remember to run:
% latex bibtex latex latex
% to resolve all references

% use this to make sure that columns are balanced on the last page.
\balancecolumns

\end{document}
